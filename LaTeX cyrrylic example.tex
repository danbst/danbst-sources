\documentclass{article}
\usepackage[utf8]{inputenc}
\usepackage[russian]{babel}
\usepackage{amssymb}
\begin{document}
%Отже, початок тексту%  
10.4 Дифференциалы высших порядков
%Значить, у глави, розділу, частини і тп є номер. Я вважаю, що повинен також бути ідентифікатор, а номер - людино-читабельний варіант цього ідентифікатора. Тобто, є зкритий механізм індексації (відносно зкритий) і відкритий - для людей. Доступ до об"єкта можна отримати безпосередньо по ідентифікатору і дуже просто і дуже швидко - головне знати ідентифікатор. Якщо ідентифікатор не відомий, тоді доведеться використати ШІ для відгадування бажання користувача.%

%Друге. Кожне слово повинно бути присутнім у внутрішньому словнику. В даному випадку - "Дифференциалы" -> "дифференциал", "высших" -> "высший", "порядков" -> "порядок". Повинен бути доступ до значення кожного слова, при цьому повинна бути пріоритетизація по змісту та категорії, по тегам. Тобто диференціал буде означати похідну, а не коробку передач.%
%Тому кожен об"єкт буде мати цілий набір тегів, в яких буде вказано все - і зміст, і суть, і категорія і тптпт. Важливо те, що теги можуть додаватись для зв"язування між текстом. Або додаватись тільки до групи об"єктів - по наявності тега можна визначити множину однотипних об"єктів. Будь-який об"єкт неможливо повністю описати, оскільки для цього знадобиться нескнченна кількість місця, але спробувати варто. Треба просто не забувати, навіщо тег додається і на що він буде впливати. Також треба намагатись передбачити  можливе використання даного тега.%
%І ще одне тоді вадливе питання - які теги є основними і повинні бути у всіх об"єктів. Не знаю, треба буде методом проб та помилок визначити. Це один з можливих напрямків створення ітеративної розширюваності, оскільки поняття "тег" є надзвичайно простим і одночасно дуже потужним.%

%Об"єкти можуть складатись (і складаються) з інших об"єктів. Кожен об"єкт має тип, хоча можливо краще не вказуват тип, а вказувати тег типу.%

%Поділення об"єкту на інші об"єкти не є єдиним, а є нескінченним (навіть рекурисвним).%

%Існують атоми. Атоми - об"єкти з тегом АТОМ (або типом АТОМ). Робота з атомами відрізяється від роботи з іншими об"єктами, оскільки задачі перекладаються з об"єкту на інформаційну систему.%

%Тег може мати значення. Значення може бути атомом, іншим об"єктом або тегом. Альтернативності створюються об"єктами з різними тегами.%

%Зв"язки також є об"єктами%

В настоящем пункте мы для удобства будем иногда вместо символа дифференцирования $d$ писать букву $\delta$, т.е. вместо $dy,dx$ писать $\delta y,\delta x$.
\end{document}
